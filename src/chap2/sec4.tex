%-------------------------- DEFINIÇÕES --------------------------
\newtheorem{def:envoltoria conica}[def:conjunto convexo]{Definição}
\newtheorem{def:independencia conica}[def:conjunto convexo]{Definição}

%----------------------------- PROPOSIÇÕES ----------------------------
\newtheorem{prop:redundancia conica}[prop:combinação convexa]{Proposição}

%------------------------------ TEOREMAS ----------------------------
\newtheorem{thm:caratheodory cones}[thm:caratheodory]{Teorema}

%---------------------------- COROLÁRIOS -----------------------------
\newtheorem{cor:caratheodory cones}[cor:caratheodory]{Corolário}

\section{Envoltória Cônica}

Na seção anterior, estudamos sobre a envoltória convexa de um conjunto de
pontos. Nesta, estudaremos o cone gerado a partir de um conjunto de direções
em $\mathbb{R}^d$, chamado de envoltória cônica ou fecho cônico. Muitos dos
resultados provados aqui serão análogos também àqueles da seção anterior.

\subsection{Envoltória e Independência Cônica}

\begin{def:envoltoria conica}
  Seja $D$ um conjunto de direções em $\mathbb{R}^n$. A envoltória cônica de
  $D$, denotado por $\cone{D}$, é o conjunto de todas as combinações cônicas
  dos elementos de $D$.
\end{def:envoltoria conica}

\begin{def:independencia conica}
 Um conjunto $D$ é cônico-independente se nenhum de seus elementos poder ser
 dado como combinação cônica dos demais
\end{def:independencia conica}

A seguir, um fato semelhante ao levantado também na seção anterior sobre
redundâncias no conjunto de geradores de um cone convexo.
\begin{prop:redundancia conica}
  Se $D = \{d_1, \dots, d_m\}$ é um conjunto de direções, e
  $D' = \{d_1, \dots, d_k\}$ é um subconjunto conico-independente
  de $D$ com $k < m$, então
  \[
    \cone{D'} = \cone{D}
  \]
  \begin{proof}
    O argumento é análogo ao mostrado na prova da proposição \ref{prop:redundancia}
  \end{proof}
\end{prop:redundancia conica}

\subsection{Teorema de Carathéodory para Cones}

Nesta parte, iremos demonstrar uma versão do Teorema de Carathéodory para os
cones convexos

\begin{thm:caratheodory cones}
  Seja $D = \{\vec d_1, \dots, \vec d_k\}$ um conjunto de direções em $\mathbb{R}^n$. Se $\vec{d} \in D$, então
  $\vec{d}$ pode ser dado como combinação cônica de até $n$ direções
  de $D$

  \begin{proof}
   Vamos provar por indução em $k$.

   Base: Para $k=1$ é trivial.

   Hipótese de Indução: Suponha que o teorema vale para todo $k$. Disso decorre
   dois casos

   I - Se $k \leq n$, não há nada a ser demosntrado

   II - Se $k > n$, então $D$ é linearmente dependente, o que implica
   \[
     \exists \mu_1, \dots, \mu_{k} \in \mathbb{R}^n, \left(\sum_{i=1}^{k}\mu_i \vec{d}_i = 0\right)
   \]
   Seja $\theta \in (0, 1)$ tal que
   \[
     \theta \sum_{i=1}^{k} \mu_i \vec{d}_i = 0
   \]
   ao passo que
   \[
     \vec{d} = \sum_{i=1}^{k}\lambda_i\vec{d}_i
   \]
   o que implica
   \[
     \vec{d} = \sum_{i=1}^{k}(\lambda_i - \theta \mu_i)\vec{d}_i
   \]
   A estratégia aqui será a mesma daquela usada para provar o Teorema de
   Carathéodory: escolha $\theta$ como o menor $\frac{\lambda_j}{\mu_j}$ tal que
   $j \in J$ onde $J$ é o conjunto dos índices de 1 até $k$ para os quais
   $\mu_k > 0$. Dessa forma, $\theta > 0$, e se seguem dois casos:

   I - $\mu_i < 0$

   Então $-\theta \mu_i > 0 \Rightarrow \lambda_i - \theta \mu_i > 0$

   II - $\mu_i > 0$

  Então $-\theta \mu_i < 0$, mas uma vez que
  \[
    \lambda_i - \theta \mu_i = \mu_i \left(\frac{\lambda_i}{\mu_i} - \theta\right)
  \]
  e pela nossa escolha de $\theta$, teremos que $\lambda_i - \theta \mu_i > 0$.

  Se $j^*$ for o índice tal que $\theta = \frac{\lambda_{j^*}}{\mu_{j^*}}$,
  então para quando $i = j^*$ temos $\lambda_i - \theta \mu_i = 0$, e
  \[
    \vec x = \sum_{i=1}^{j^*-1} (\lambda_i - \theta \mu_i) \vec d_i + \sum_{i=j^*+1}^{k} (\lambda_i - \theta \mu_i) \vec d_i
  \]
  Logo $\vec d$ pode ser escrito como combinação cônica de $k-1$ direções e,
  pela hipotese de indução, pode ser então escrito como combinação de até
  $n$ direções. Portanto, pelo Princípio de Indução, o teorema vale para
  qualquer conjunto de direções.
  \end{proof}
\end{thm:caratheodory cones}

\begin{cor:caratheodory cones}
  Seja $D$ um conjunto de direções em $\mathbb{R}^n$. Se $\vec{x} \in \cone{D}$, então
  $\vec{x}$ pode ser dado como combinação cônica de até no máximo $n$ direções
  extremas de $\cone{D}$.
\end{cor:caratheodory cones}

O corolário anterior é ainda mais interessante no caso em que o cone convexo
é finitamente gerado. Se $D$ é finito, então qualquer direção de
$\cone{D}$ pode ser gerada por até $n$ direções extremas de $\cone{D}$, o
que é equivalente a dizer que pode ser gerado por até $n$ direções de $D$.
Se $D$ for cônico-independente, então ele é o conjunto das direções
extremas de $\cone{D}$.

