% -------------------------- DEFINIÇÃO --------------------
\newtheorem{def:convex hull}[def:conjunto convexo]{Definição}
\newtheorem{def:independencia convexa}[def:conjunto convexo]{Definição}
\newtheorem{def:simplex}[def:conjunto convexo]{Definição}
\newtheorem{def:combinação afim}[def:conjunto convexo]{Definição}

% ------------------------- PROPOSIÇÃO --------------------
\newtheorem{prop:redundancia}[prop:combinação convexa]{Proposição}
\newtheorem{prop:pontos extremos na fronteira}[prop:combinação convexa]{Proposição}
\newtheorem{prop:conjuntos convexos fechados}[prop:combinação convexa]{Proposição}
\newtheorem{prop:conjuntos convexos limitados}[prop:combinação convexa]{Proposição}

% ------------------------------ LEMA --------------------------
\newtheorem{lemma:afim}{Lema}[chapter]
\newtheorem{lemma:dentro ou fora}{Lema}[chapter]

%------------------------------ TEOREMA ---------------------------
\newtheorem{thm:caratheodory}{Teorema}[chapter]
\newtheorem{thm:conjuntos convexos compactos}[thm:caratheodory]{Teorema}

% ---------------------------- COROLÁRIO -----------------------
\newtheorem{cor:caratheodory}{Corolário}[chapter]

\section{Geometria dos Poliedros Convexos}

Nesta seção iremos obsevar os poliedros convexos a partir de sua
natureza geométrica ao mesmo tempo que mostramos sua equivalência com
os conceitos algébricos vistos nas seções anteriores.

\subsection{Espaços Afim}

Antes de ir mais afundo na natureza geométrica dos conjuntos
poliédricos, iremos recuar um pouco e observar alguns conceitos
algébricos. Relembrando, uma das noções primárias da Álgebra Linear é a de
independência linear, em que dizemos que um conjunto de vetores
é linearmente independente se nenhum vetor poder ser dado
como combinação linear dos demais. Uma classe de combinações
lineares que nos é últil são as chamadas combinações afim

\begin{def:combinação afim}
Sejam $\vec v_1, \ldots, \vec v_k$ vetores do $\mathbb{R}$. $u$ é uma
combinação afim dos vetores citados  se existem $\lambda_1, \ldots,
\lambda_k$ tal que
\begin{equation*}
	\vec u = \sum_{i =1}^{k}\lambda_i \vec v_i \quad \quad \sum_{i=1}^{k} \lambda_i = 1
\end{equation*}
\end{def:combinação afim}
Combinações afim são uma especialização das combinações
lineares, visto a restrição de que a soma dos escalares
que acompanham os vetores deve ser igual a 1. Observemos
também que as combinaçãoes convexas, por sua vez, são uma
especialização das combinações afim, em que adiciona-se a
restrição de que os pesos devem ser positivos.

As combinações lineares afim de um conjunto de vetores
formam um conjunto fechado chamado de \textbf{subsespaço afim},
que são análogos aos hiperplanos lineares a menos
da restrição de conterem a
origem do $\mathbb{R}^n$.

% imagem de compração de um hiperplano linear
% e um afim

Os hiperplanos  afins, de forma análoga ao lineares, podem ser
definidos a partir de um sistema linear na forma $A\vec x = b$
ou através de um conjunto de no mínimo $n$ geradores.
Um conjunto de vetores em que pelo menos um dos membros
é combinação afim dos demais é chamado de afimmente dependente.
Geometricamente, isso significa que pelo menos um dos vetores estará
no espaço afim gerado pelos demais.

É sabido que qualquer conjunto com mais de $n$ vetores em
$\mathbb{R}^n$ é linearmente dependente, mas qual será o número mínimo
de vetores em um conjunto para afirmamos com certeza que eles são
afimmente dependentes?

A envoltória convexa de dois pontos é um segmento de reta, e,
por sua vez, o subsespaço afim gerado por esses mesmos pontos é a
a própria reta que eles definem, ou seja, obtemos o subespaço
afim gerado ao extender infinitamente o segmento de reta em
ambos os seus sentido. Já três pontos não colineares
definem um triângulo, que é uma figura plana, e o subsespaço afim gerado por
esses três pontos é justamnte o plano que contém esse triângulo; é como se
tivessemos expandido infinitamente a área do triângulo assim como fizemos
com o comprimento do segmento de reta anteriormente. Por conseguinte,
para obter o \textit{span} afim a partir de uma envoltória convexa, basta
extendemos infinitamente a envoltória nas direções possíveis.

Isso nos permite agora responder a questão feita mais atrás:
expandir infinitamente um simplex de dimensão $n$ que é gerado
por $n+1$ vetores do  $\mathbb{R}^n$,
, é suficiente para obtermos todo o $\mathbb{R}^n$
e qualquer outro vetor do $\mathbb{R}^n$ pode ser dado como
combinação afim dos pontos extremos do simplex que fora expandido.
Portanto, a conclusão que chegamos é que qualquer conjunto com mais
de $n+1$ vetores é necessariamnte afimmente independente. Esse resultado
é provado formalmente com uma abordagem mais algébrica no lema a seguir.

\begin{lemma:afim}
	\label{lemma:afim}
	Se $P = \{\vec p_1, \ldots, \vec p_k\} \subset \mathbb{R}^n$ é um conjunto
	finito de pontos com $k > n + 1$, então existem $\mu_1, \ldots, \mu_n
	\in \mathbb{R}$ não todos nulos tal que
	\begin{equation*}
		\displaystyle\sum_{i=1}^k \mu_i \vec p_i = 0 \quad\quad \displaystyle\sum_{i=1}^k \mu_i = 0
	\end{equation*}
	e $P$ é um conjunto afimmente dependente.

	\begin{proof}
		Da Álgebra Linear, sabemos que se $\#P = k > n$, então $P$ é um conjunto
		linearmente dependente, o que implica dizer que há $\mu_1, \ldots, \mu_n
		\in \mathbb{R}$ não todos nulos tal que
		\[\displaystyle\sum_{i=1}^k \mu_i \vec p_i = 0\]

		Além disso, observemos que o conjunto $\{\vec p_2 - \vec p_1, \ldots, \vec p_k - \vec p_1\}$
		também é linearmente dependente, pois contém mais de $n$ vetores,
		e também que
		\begin{gather*}
		  \sum_{i=2}^{k} \mu_i (\vec p_i - \vec p_1) \\
		  \sum_{i=2}^{k} \mu_i \vec p_i - \vec p_1 \sum_{i=2}^{k} \mu_i \\
		  - \mu_1 \vec p_1 - \vec p_1 \sum_{i=2}^{k} \mu_i \\
		\end{gather*}
		Dessa forma, se tivermos que
		\[
		  \mu_1 = - \sum_{i=2}^{k} \mu_i
		\]
		então é verdade que
		\[
		  \sum_{i=2}^{k} \mu_i (\vec p_i - \vec p_1) = 0
		\]
		e também
		\[
		  \sum_{i=1}^{k} \mu_i = 0
		\]

		Resta mostrar que o conjunto $P$ é afimmente dependente. Para isso, suponha
		que $P$ esteja ordenado de forma que para todo $i \in I = \{1, \ldots, k\}$
		é verdade que $\mu_1 \geq \mu_i$. Obeservamos então que
	  \begin{gather*}
	    \mu_1 \vec p_1 + \mu_2 \vec p_2 + \ldots + \mu_k \vec p_k = 0 \\
        \mu_2 \vec p_2 + \ldots + \mu_k \vec p_k = - \mu_1 \vec p_1 \\
	  \end{gather*}
	  Dividindo a equação acima toda por $-\mu_1$ otbemos
	  \begin{gather*}
	    \sum_{i=2}^{k} -\frac{\mu_i}{\mu_1} \vec p_i = \vec p_1
	  \end{gather*}
	  Mas também
	  \begin{gather*}
        \sum_{i=2}^{k} \mu_i = -\mu_1 \\
	    \sum_{i=2}^{k} -\frac{\mu_i}{\mu_1} = 1
	  \end{gather*}
	  Portanto $p_1$ pode ser dado como combinação afim dos demais pontos de
	  $P$, e o conjunto dos pontos é afimmente dependente.
	\end{proof}
\end{lemma:afim}

\subsection{Envoltória Convexa}

\begin{def:convex hull}
	\label{def:convex hull}
	Seja $P$ um conjunto de pontos em $\mathbb{R}^n$. Chamamos de envoltória
	convexa $P$, ou fecho convexo, o conjunto
	$\conv{P}$ dado por todas as combinações convexas dos pontos de $P$
\end{def:convex hull}

Em outras referências, pode-se encontrar uma definição equivalente de que a
envoltória convexa, é o menor conjunto convexo que contém todos os pontos de $P$.
Neste texto, usaremos a definição destacada por ela ser mais algebricamente fecunda.

\begin{def:independencia convexa}
	Seja $P \subset \mathbb{R}^n$ é um conjunto de pontos. $P$ é convexo-independente
	se nenhum de seus pontos pode ser dado como combinação convexa de outros dois.
\end{def:independencia convexa}

A definição acima especializa o conceito de independência linear para conjuntos convexos,
que nos será uma definição útil. Dessa forma, um conjunto convexo-dependente será aquele
no qual há pelo menos um ponto que pode ser dado como combinação convexa de outros dois.
Outra especialização que faremos será do conceito de geradores: se um conjunto convexo
$X$ é a envoltória convexa de um conjunto $P$, então $P$ é um conjunto gerador de $X$.
Ademais, se $P$ é convexo-independente, então os pontos de $P$ são os pontos extremos
de $X$, o que especializa o conceito de base da álgebra linear para os pontos extremos.

Na Teoria de PL, nosso interesse irá se concentrar nos conjuntos convexo que são
finitamente gerados, isto é, que podem ser determinados por um conjunto finito
de pontos geradores. Exemplo de conjuntos convexos finitamente gerados são
os poliedros tridimensionais e os polígonos, enquanto que o círculo, ou a esfera,
são exemplos de conjuntos convexos que não são finitamente gerados.

\begin{figure}[h]
\centering
% Primeira figura: Polígono Convexo
\begin{subfigure}{0.45\textwidth}
	\centering
	\begin{tikzpicture}
		\draw[thick, fill=blue!20] (0,0) -- (2,1) -- (3,3) -- (1.5,4) -- (-1,3) -- cycle;
	\end{tikzpicture}
	\caption{Conjunto convexo finitamente gerado}
	\label{fig:poligono}
\end{subfigure}
\hfill
% Segunda figura: Círculo
\begin{subfigure}{0.50\textwidth}
	\centering
	\begin{tikzpicture}
		\draw[thick, fill=red!20] (0,0) circle (2);
	\end{tikzpicture}
	\caption{Conjunto convexo não finitamente gerado}
	\label{fig:circulo}
\end{subfigure}
\caption{Comparação entre um polígono convexo e um círculo.}
\end{figure}

\begin{prop:redundancia}
	Seja $P = \{p_1, \ldots, p_m\}$ um conjunto convexo dependente e
	$P' = \{p_1, \ldots, p_k\}$ um subconjunto convexo-independente
	de $P$ com $k < m$. Então é correto dizer que
	\begin{equation*}
		\conv{P'} = \conv{P}
	\end{equation*}
\end{prop:redundancia}

Dado um conjunto de geradores, é fácil observar que, ao obtermos um
novo conjunto sem as redundâncias, então a envoltória convexa, como
foi definida em \ref{def:convex hull}, será exatamente a mesmas para
ambos os conjuntos. Além disso, se $P$ for um conjunto convexo
independente, então seus elementos são os pontos extremos de \(\conv{P}\),
como fora observado mais previamente.

Uma classe especial de envoltórias convexas são os \textbf{simplex}
A ideia geométrica por detrás do conceito de simplex é do ``mais simples
conjunto convexo num espaço euclideano de dimensão finita''. Por exemplo:
o menor número de pontos necessários para delimitar uma região no plano é
3, e a região delimitada por eles é um triângulo. Já em dimensão 3, para
delimitar uma região no espaço são necessários ao menos quatro pontos,
que definirão um tetraedro. Generalizando essa primitiva, obtemos a definição.

\begin{def:simplex}
	Chama-se simplex a envoltória convexa de um conjunto de $n+1$ pontos
	em $\mathbb{R}^n$ afimmente independetes.
\end{def:simplex}

A razão por detrás da restrição de que os pontos devem ser afimmente independentes
é porque eles não podem estar todos contidos num mesmo hiperplano afim, de modo
que o conjunto convexo gerado por eles tenha dimensão igual ao do $\mathbb{R}^n$.

\begin{figure}[h]
	\centering
	% Primeira figura: Triângulo
	\begin{subfigure}{0.45\textwidth}
		\centering
		\begin{tikzpicture}
			\draw[thick, fill=blue!20] (0,0) -- (3,0) -- (1.5,2.5) -- cycle;
		\end{tikzpicture}
		\caption{Simplex em 2D}
		\label{fig:triangulo}
	\end{subfigure}
	\hfill
	% Segunda figura: Tetraedro em 3D
	\begin{subfigure}{0.45\textwidth}
		\centering
		\begin{tikzpicture}
			% Definição dos vértices
			\tdplotsetmaincoords{70}{120}
			\begin{scope}[tdplot_main_coords]
				% Arestas do tetraedro
				\draw[thick, fill=red!20, opacity=0.7] (0,0,0) -- (2,0,0) -- (1,1.5,1.5) -- cycle;
				\draw[thick, fill=red!30, opacity=0.7] (0,0,0) -- (1,1.5,1.5) -- (0,2,0) -- cycle;
				\draw[thick, fill=red!40, opacity=0.7] (2,0,0) -- (1,1.5,1.5) -- (0,2,0) -- cycle;
				\draw[thick] (0,0,0) -- (2,0,0) -- (0,2,0) -- cycle;
			\end{scope}
		\end{tikzpicture}
		\caption{Simplex em 3D}
		\label{fig:tetraedro}
	\end{subfigure}
	\caption{Comparação entre simplex de dimensões diferentes.}
\end{figure}

\subsection{Teorema de Carathéodory}

O Teorema de Carathéodory é um dos resultados mais fundamentais da geometria
convexa, pois ele nos permite dizer que um ponto qualquer em um conjunto
convexo pode ser gerado por um número finito de outros pontos nesse conjunto,
mas não somente isso como também o limitante do tamanho desse conjunto de
geradores.

\begin{thm:caratheodory}[Carathéodory]
	Seja $P \subset \mathbb{R}^n$ um conjunto finito de pontos. Se $\vec x \in \conv{P}$, então
	  $\vec x \in \conv{P'}$ para algum $P' \subset P$ com cardinalidade igual a $n + 1$.

	\begin{proof}
		Com efeito, se $\vec x \in \conv{P}$, e $\#P = k$, então existem $\lambda_1, \ldots, \lambda_k
		\in \mathbb{R}$ com $\lambda_i \in [0, 1]$ tal que
		\begin{equation}
		\label{eq_thm_caratheory}
		\begin{gathered}
			x = \displaystyle\sum_{i=1}^k \lambda_i \vec p_i \\
			\displaystyle\sum_{i=1}^k \mu_i = 1
		\end{gathered}
		\end{equation}
		Se $k \leq n + 1$, então nada há a demonstrar, do contrário, o lema \ref{lemma:afim}
		garante que existem $\mu_1, \ldots, \mu_n \in \mathbb{R}$ não todos nulos tal que
		\[
		  \displaystyle\sum_{i=1}^k \mu_i \vec p_i = 0 \quad\quad \sum_{i=1}^{k} \mu_i = 0
		\]
		e como \(\alpha \displaystyle\sum_{i=1}^k \mu_i \vec p_i = 0\) para todo $\alpha \in \mathbb{R}$
		então podemos rescrever a equação \ref{eq_thm_caratheory} como
		\begin{gather*}
			\vec x = \displaystyle\sum_{i=1}^k \lambda_i \vec p_i - \alpha \displaystyle\sum_{i=1}^k \mu_i \vec p_i \\
			\vec x = \displaystyle\sum_{i=1}^k (\lambda_i - \alpha \mu_i) \vec p_i
		\end{gather*}
		donde temos que
		\[\displaystyle\sum_{i=1}^k (\lambda_i - \alpha \mu_i) = 1\]

		Agora iremos tentar escrever $\vec x$ como uma combinação convexa de até $n + 1$ pontos.
		Para atingir esse objetivo, escolhemos $\alpha$ tal que
	    \[
	      \alpha = \min_{i \in \{1, \dots, k\}} \left\{\frac{\lambda_i}{\mu_i}\ |\ \mu_i > 0\right\}
	    \]

		Logo, $\alpha > 0$, e para $i \in \{1, \ldots, k\}$, temos dois casos.

		I - $\mu_i \geq 0$

		Temos que
		\[\lambda_i - \alpha \mu_i = \mu_i \left(\frac{\lambda_i}{\mu_i} - \alpha\right)\]
		e pela nossa escolha de $\alpha$, então \(\lambda_i - \alpha \mu_i \geq 0\)

		II - $\mu_i < 0$

		Uma vez que $\alpha > 0$, então $\alpha \mu_i < 0$ e, portanto,
		\(\lambda_i - \alpha \mu_i \geq 0\)

		digamos que $j^*$ seja tal que $\alpha = \frac{\lambda_{j^*}}{\mu_{j^*}}$. Observe que
		$\lambda_{j^*} - \alpha \mu_{j^*} = 0$, e podemos expressar $x$ como
		\[\vec x = \displaystyle\sum_{i=1}^{j^* - 1} (\lambda_i - \alpha \mu_i)\vec p_i +
				\displaystyle\sum_{i=j^* + 1}^{k} (\lambda_i - \alpha \mu_i)\vec p_i\]
		donde concluímos que $x$ pode ser escrito como combinação convexa de $k - 1$ pontos de $\conv{P}$

		Como podemos repetir esse processo enquanto $k > n + 1$, ou seja, enquanto o lema \ref{lemma:afim}
		puder ser aplicado, então $\vec x$ pode ser escrito como combinação convexa de até $n + 1$ pontos de $\conv{P}$.
	\end{proof}
\end{thm:caratheodory}

Se $P \subset \mathbb{R}^n$ é um conjunto
convexo-independente, então para qualquer $\vec x \in \conv{P}$, $\vec x$ pode ser dado
como combinação convexa de até $n+1$ pontos de $P$, ou seja, de até $n+1$ pontos extremos.
Formalizemos essa conclusão como corolário do Teorema de Carathéodory.

\begin{cor:caratheodory}
	Seja $P \subset \mathbb{R}^n$ um conjunto convexo-independente. Se $\vec x \in \conv{P}$
	então $\vec x$ pode ser dado como combinação convexa de até $n + 1$ pontos de $P$
\end{cor:caratheodory}

Em outras palavras, o corolário afirma que um ponto qualquer num fecho convexo
pode ser dado como combinação convexa de até $n+1$ pontos extremos.

\subsection{Topologia de Evoltórias Convexas}

Nesta seção iremos analisar algumas propriedades topológicas básicas de
interesse sobre a envoltória convexa de um conjunto de pontos. Mas não se
preocupe, apesar de usarmos conceitos de topologia, iremos apresentá-los
e usá-los de forma bem elementar a fim de não desviar muito dos nossos
objetios e complicar em demasiado nossas construções, já que está fora do
escopo desse texto as noções mais aprofundadas desse tema.

\begin{prop:pontos extremos na fronteira}
Seja $P$ um conjunto convexo. Se $\bar{\vec{x}}$ é um ponto extremo de $P$,
então $\bar{\vec{x}}$ é um ponto na fronteira de $P$.

  \begin{proof}
    Para provarmos esse resultado, basta mostrarmos que para qualquer bola
    aberta $B$ centrada em $\bar{\vec{x}}$, $B$ intercepta pontos tanto de
    $P$ quanto de $\mathbb{R}^n / P$. Suponha por contradição que haja uma
    bola aberta centrada em $\bar{\vec{x}}$ tal que $B \subsetneq P$. Isso
    implica que $\bar{\vec{x}}$ está no interior de um segmento de reta
    inteiramente contido em $P$, e não é, pois, ponto extremo.
  \end{proof}
\end{prop:pontos extremos na fronteira}

Antes de partirmos para o próximo resultado, elucidemos um fato. Suponha que
$X$ seja um conjunto convexo no $\mathbb{R}^n$ gerado por pontos de $P$.
Para todo ponto $\vec x$ somente uma das afirmações pode ser verdadeira:

\begin{enumerate}
  \item $\vec x$ está em $X$.
  \item Existe um hiperplano com equação $\transp a \vec y = b$ tal que
  \[
    \transp a \vec x > b \text{ e } \transp a \vec q < b
  \]
  para todo $\vec q \in X$
\end{enumerate}

A segunda afirmação é mais forte e ela é consequência do Teorema do
Hiperplano Separador, um importante resultado em geometria convexa, mas
o qual não provaremos aqui. Dessa forma, temos duas opções, dado um ponto
e um conjunto convexo, então ou o ponto está no conjunto, ou existe um
hiperplano separando os dois. Com isso em mente, iremos agora provar
que conjuntos convexos finitamente gerados são fechados.

\begin{prop:conjuntos convexos fechados}
Se $C$ é a envoltória convexa de $P$, então $C$ é fechado.
  \begin{proof}
    O que faremos na verdade será mostrar que se $\vec x$ não é um ponto de $X$,
    então ele não poderia ser um ponto de fronteira.

    Se $\vec x \notin X$, então existe um hiperplano $H$ com
    equação $\transp a \vec y = b$ que divide
    o $\mathbb{R}^n$ em dois meio-espaços abertos que separam
    $\vec x$ e $X$. Digamos que $\vec x \in H^+$ e que $X \subset H^-$.
    Supnha que $\vec z$ seja um ponto no hiperplano $H$ tal que
    $|| \vec z - \vec x ||$ seja a mínima distância entre $\vec x$
    e o hiperplano $H$.

    Se $\alpha = || \vec z - \vec x ||$,
    então temos que para toda bola aberta centrada em $\vec x$
    e com raio manor que $\alpha$ estará totalmente contida em
    $H^+$. Como meios-espaços são conjuntos convexos, e $X \subset H^-$,
    então existe uma bola aberta centrada em $\vec x$ que não
    contém nenhum ponto de $X$, e $\vec x$ não poderia ser
    um ponto de fronteira de $X$.
  \end{proof}
\end{prop:conjuntos convexos fechados}

\begin{prop:conjuntos convexos limitados}
  Se $C = \conv{P}$ é a envoltória convexa de um conjunto finito $P$ de pontos,
  então $C$ é limitado.

  \begin{proof}
   Com efeito, uma vez que $P$ é finito, então deve have uma bola aberta que
   contém todos os seus pontos e, por conseguinte, a envoltória convexa desse
   conjunto, portanto $C = \conv{P}$ é limitado.
  \end{proof}
\end{prop:conjuntos convexos limitados}

Para fins de síntese, reunamos as proposiçõe anteriores no teorema a seguir

\begin{thm:conjuntos convexos compactos}
  Se $C = \conv{P}$ é a envoltória convexa de um conjunto finito de pontos,
  então $C$ é compacto.\footnote{Chama-se de compactos os conjuntos que são
  simultâneamente fechados e limitados.}
\end{thm:conjuntos convexos compactos}
