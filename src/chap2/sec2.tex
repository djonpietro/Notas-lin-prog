%==================== Definições =======================
\newtheorem{def:ponto extremo}[def:conjunto convexo]{Definição}

\newtheorem{def:raio}[def:conjunto convexo]{Definição}

\newtheorem{def:direção}[def:conjunto convexo]{Definição}

\newtheorem{def:direção extrema}[def:conjunto convexo]{Definição}

\newtheorem{def:ponto degenerado}[def:conjunto convexo]{Definição}

\newtheorem{def:face}[def:conjunto convexo]{Definição}


%================== Proposições ================
\newtheorem{prop:combinação convexa}{Proposição}[chapter]

\newtheorem{prop:hiperplano e ponto extremo}[prop:combinação convexa]{Proposição}

\newtheorem{prop:direção}[prop:combinação convexa]{Proposição}

\newtheorem{prop:aresta}[prop:combinação convexa]{Proposição}

%=================== Teoremas ===================

\newtheorem{thm:ponto extremo}{Teorema}[chapter]

\section{Pontos Extremos, Direções e Faces}

\subsection{Pontos Extremos}

O conceito de ponto extremo de um conjunto convexo $\mathbb{V}$ é uma
especialização do conceito de ``independência linear'' para combinações
lineares convexas.

\begin{def:ponto extremo}
	Um vetor $x$ num conjunto convexo $\mathbb{V}$ é chamado ponto extremo de
	$\mathbb{V}$ se ele não pode ser expresso como combinação linear
	convexa de nenhum outro subconjunto de $\mathbb{V}$.
\end{def:ponto extremo}

De grosso modo, podemos dizer que os pontos extremos de um conjunto convexo
é um conjunto linearmente independente se considerarmos apenas suas combinações
convexas. Além disso, é possível ver também que os pontos extremos formam
uma ``base'' para conjuntos convexos, ou seja, qualquer ponto de um conjunto
convexo pode ser dado como combinação convexa dos pontos extremos, mas não se
preocupe com isso agora, pois iremos explorar essa ideia mais adiante.

Agora, iremos explorar mais o significado geométrico de um ponto extremo.
Mas para isso, iremos demonstrar um resultado que irá nos auxiliar nesta tarefa.

\begin{prop:combinação convexa}
	\label{prop:combinação convexa}
	Se $\alpha$ e $\beta$ são números reais e $\gamma$ é uma combinação linear
	estritamente convexa desses números, tal que $\alpha \leq \gamma$ e
	$\beta \leq \gamma$, então \[\alpha = \beta = \gamma\]

	\begin{proof}
		Primeiros vamos explicitar a ideia geométrica de uma combinação linear
		convexa para números reais. Tradicionalmente, o conjunto dos reais é
		compreendido como uma reta cujos pontos estão, respectivamente, associados
		a um número real. Dessa forma, o conjunto das combinações lineares convexas
		entre dois pontos da reta real é o segmento de reta que une esses dois pontos.

		Dito isso, se $\gamma$ é uma combinação convexa de $\alpha$ e $\beta$, então é verdade que
		\begin{equation}
			\label{eq:prop1}
			\alpha \leq \gamma \leq \beta
		\end{equation}uma vez que o ponto associado a $\gamma$ deve estar
		disposto no segmento unindo os pontos associados a $\alpha$ e $\beta$.
		Visto que, por hipótese
		\begin{align*}
			\alpha \leq \gamma \quad \beta \leq \gamma
		\end{align*}
		então, da expressão à direita e de (\ref{eq:prop1})
		segue que $\beta = \gamma$, ao passo que
		\begin{gather*}
			\lambda \alpha + (1 - \lambda) \gamma = \gamma \\
			\lambda \alpha = \lambda \gamma \\
			\alpha = \gamma
		\end{gather*}
	\end{proof}
\end{prop:combinação convexa}

Essa proposição, num primeiro momento, parece estar desconectada da ideia dos
pontos extremos, mas apesar de simples, ela é fundamental para construirmos
a geometria daquele conceito. Vamos enunciar mais uma proposição para
seguirmos com o nosso objetivo. Em resumo, ela diz que, se um ponto em um
hiperplano é combinação convexa de outros dois que compartilham o mesmo
semiespaço, então esses dois últimos também estão no hiperplano em questão.

\begin{prop:hiperplano e ponto extremo}
	\label{prop:hiperplano e ponto extremo}
	Sejam os pontos $\vec{x}'$ e $\vec{x}''$ pertencentes a um mesmo
	semiespaço $H^-$. Se uma combinação estritamente convexa desses pontos
	está no hiperplano $H$, então ambos os pontos pertencem a esse hiperplano.

	\begin{proof}
		Seja $\bar{\vec{x}}$ uma combinação estritamente convexa de $\vec{x}'$ e $\vec{x}''$ tal que
		\[\lambda \vec{x}' + (1 - \lambda) \vec{x}'' = 		\bar{\vec{x}}\]
		Se $H = \{\vec{x}\ |\ \transp{a}\vec{x} = b\}$ e $\bar{\vec{x}} \in H$, então
		\[\transp{a}\bar{\vec{x}} = b\]
		e disso segue que
		\begin{equation}
			\label{eq:prop2}
			\lambda (\transp{a}\vec{x}') + (1 - \lambda) 	(\transp{a}\vec{x}'') = b
		\end{equation}
		Digamos que $H^-$ seja definido como
		\[H^- = \{\vec{x}\ |\ \transp{a}\vec{x} \leq b\}\]
		Com efeito, se $\vec{x}'$ e $\vec{x}''$ estão em $H^-$, então
		\[\transp{a}\vec{x}' \leq b \quad \transp{a}\vec{x}'' \leq b\]


		Como bem sabemos, o resultado do produto escalar entre vetores é um
		número real, e o que a expressão \ref{eq:prop2} está nos mostrando é
		que $b$ é um combinação convexa dos números reais $\transp{a}\vec{x}'$
		e $\transp{a}\vec{x}''$. Ora, mas também temos que
		$\vec{x}', \vec{x}'' \in H^-$, o que implica pela Proposição
		\ref*{prop:combinação convexa} que
		\[\transp{a}\vec{x}' = b \quad \transp{a}\vec{x}'' = b\]
		ou seja, $\vec{x}', \vec{x}'' \in H$
	\end{proof}
\end{prop:hiperplano e ponto extremo}

Com esse último resultado em mãos, agora iremos caracterizar os pontos extremos
a partir das noções já vistas de semiespaços e hiperplanos. Para viés de
síntese, sendo $X = \{\vec{x}\ |\ A\vec{x} = \vec{b},\ \vec{x} \geq 0\}$,
com $A$ sendo $m \times n$, iremos chamar os $n + m$ hiperplanos associados aos
semiespaços definidores de $X$ (as $n$ restrições agrupadas em $A$ mais
a $m$ restrições de não negatividade) de \textbf{hiperplanos definidores} de $X$.

\begin{thm:ponto extremo}
	\label{thm:ponto extremo}
	Um vetor $\bar{\vec{x}}$ é ponto extremo de um conjunto poliédrico
	$X = \{\vec{x}\ |\ A\vec{x} = \vec{b},\ \vec{x} \geq \vec{p}\}$, com
	$A \in \mathbb{R}^{m \times n}$, se $\bar{\vec{x}}$ pertence a pelo menos $n$
	hiperplanos linearmente independentes que definem $X$.

	\begin{proof}
		Suponha por contradição que $\bar{\vec{x}}$ pertence a pelo menos $n$
		hiperplanos linearmente independentes definidores de $X$, mas não é um
		ponto extremo. Com efeito, $\bar{\vec{x}}$ pode ser dado como combinação
		estritamente convexa de vetores $\vec{x}'$ e $\vec{x}''$ em $X$.
		Pela Proposição \ref{prop:hiperplano e ponto extremo}, tanto
		$\bar{\vec{x}}'$ quanto $\bar{\vec{x}}''$ devem pertencer a esses $n$
		hiperplanos definidores.

		Entretanto, o sistema linear dado pelas equações desses hiperplanos é uma sistema quadrado $n \times n$ cujas linhas são linearmente independentes, o que implica que ele tem solução única. Portanto \[\vec{x}' = \vec{x}''\]logo $\bar{\vec{x}}$ não pode ser uma combinação estritamente convexa daqueles vetores, contrariando a afirmação de que ele poderia.

		Por outro lado, suponhamos agora a contrapositiva de que $\bar{\vec{x}}$ pertença a $r < n$ hiperplanos linearmente independentes, e que o sistema linear dado por eles é \[B\vec{x} = \vec{c}\]onde $B$ é uma matriz $r \times n$. O que faremos agora será construir vetores $\vec{x}'$ e $\vec{x}''$ dos quais $\bar{\vec{x}}$ é uma combinação estritamente convexa, e não poderia, pois, ser um ponto extremo.

		Por ser uma matriz retangular larga (mais colunas do que linhas), o posto de $B$ é no máximo $r$, logo ela não pode ter posto cheio, o que implica que existe um vetor $\vec{d} \neq 0$ tal que $B \vec{d} = 0$. Construímos o vetor $\vec{d}$ porque se temos um sistema com infinitas soluções e conhecemos pelos uma, podemos encontrar outras somando essa solução com uma vinda do sistema homogêneo, e usaremos essa ideia para construir $\vec{x}'$ e $\vec{x}''$.

		Digamos que
		\begin{equation*}
			\vec{x}' = \bar{\vec{x}} + \epsilon \vec{d}
			\quad
			\vec{x}'' = \bar{\vec{x}} - \epsilon \vec{d}
		\end{equation*}
		Por conseguinte temos que tanto $\vec{x}'$ quanto $\vec{x}''$ satisfazem o sistema $B\vec{x} = \vec{c}$, pertencendo aos $r$ hiperplanos nos quais $\bar{\vec{x}}$ também está.

		Para que $\vec{x}'$ e $\vec{x}''$ satisfaçam as demais $n - r$ restrições, basta tomarmos um $\epsilon > 0$ suficientemente pequeno. Isso porque, se pensarmos geometricamente,  $\vec{x}'$ e $\vec{x}''$ estão ambos nas semirretas com vértice em $\bar{\vec{x}}$ e cujas direções são dadas por $\vec{d}$ e $-\vec{d}$ respectivamente. Desse modo, para que não avancemos para além da fronteira de $X$, basta que $\epsilon$ seja pequeno o suficiente para tal.

		Por fim, escolhendo $\lambda = 0.5$, temos que
		\begin{gather*}
			0.5 \cdot \vec{x}' + 0.5 \cdot \vec{x}'' \\
			0.5 \cdot (\bar{\vec{x}} + \epsilon \vec{d}) +
			0.5 \cdot (\bar{\vec{x}} - \epsilon \vec{d}) \\
			0.5 \cdot \bar{\vec{x}} + 0.5 \cdot \bar{\vec{x}}
		\end{gather*}
		concluindo então que
		\begin{equation*}
			0.5 \cdot \vec{x}' + (1 - 0.5) \cdot \vec{x}'' = \bar{\vec{x}}
		\end{equation*}
		e encontramos $\vec{x}'$ e $\vec{x}''$ dos quais uma combinação estritamente convexa é $\bar{\vec{x}}$. Portanto se $\bar{\vec{x}}$ não pertence ao menos $n$ dos hiperplanos que definem $X$, então $\bar{\vec{x}}$ não pode ser um ponto extremo.
	\end{proof}
\end{thm:ponto extremo}

Você talvez deve estar se perguntando agora sobre os pontos extremos que pertencem a mais do que $n$ dos hiperplanos definidores. Para esses pontos usamos uma nomenclatura especial, que iremos formalizar na definição a seguir.

\begin{def:ponto degenerado}
	Seja $X = \{\vec{x}\ |\ A\vec{x} = \vec{b}, \vec{x} \geq 0\}$ um conjunto poliédrico em que $A$ é $m \times n$. Se $\vec{x}$ é um ponto extremo de $X$ tal que $\vec{x}$ pertença a mais do que $n$ dos $n + m$ hiperplanos definidores de $X$, então $\vec{x}$ é chamado de \textbf{ponto extremo degenerado}. O número excedente de hiperplanos que possuem $\vec{x}$ é chamado de \textbf{ordem de degeneração}.
\end{def:ponto degenerado}

Em resumo, podemos definir um ponto extremo com $n$ hiperplanos definidores linearmente independentes do nosso conjunto convexo $X$. Se existe mais de uma maneira de definir esse ponto extremos com $n$ hiperplanos, então esse ponto é um ponto degenerado.

%Pontos extremos são então soluções de um sistema dado por um subconjunto linearmente independente das restrições na forma padrão de um PPL. Sabemos da Álgebra Linear que em $\mathbb{R}^n$, a interseção de $n$ hiperplanos LI é um ponto, esse que também pode ser chamado de vértice. Pense bem, no plano, os hiperplanos são retas, e quando desenhamos um polígono, os vértices dele são as interseções entre duas retas. Da mesma forma, em poliedros em três dimensões, um vértice é o encontro de 3 faces, que são os hiperplanos de um espaço de dimensão 3. Portanto, o conceito algébrico de ponto extremos está diretamente atrelado ao conceito de vértice da geometria, e podemos dizer que os pontos extremos são então os vértices do poliedro dado pelo conjunto viável do PPL.

Pontos extremos de um PPL são soluções de um sistema formado por um subconjunto linearmente independente de restrições na forma padrão. Da Álgebra Linear, sabemos que, em $\mathbb{R}^n$, a interseção de $n$ hiperplanos linearmente independentes é um ponto, o qual também pode ser chamado de \textbf{vértice}.

Para visualizar isso, considere exemplos em dimensões menores:
\begin{itemize}
	\item No plano ($\mathbb{R}^2$): Os hiperplanos são retas, e ao desenharmos um polígono, seus vértices correspondem às interseções de pares de retas.
	% Figura no plano (R^2)
	\begin{figure}[H]
	\centering
	\begin{tikzpicture}
		% Desenho do polígono no plano
		\fill[blue!10] (0,0) -- (4,0) -- (3,2.5) -- (0,2) -- cycle;

		% Linhas (hiperplanos no plano)
		%\draw[thick, red] (0,2) -- (4,1.5) node[anchor=south east] {Hiperplano 1};
		%\draw[thick, green!70!black] (0,0) -- (4,3) node[anchor=south] {Hiperplano 2};
		%\draw[thick, purple] (0,3.5) -- (4,-0.5) node[anchor=north] {Hiperplano 3};

		% Polígono (interseção das restrições)
		\draw[thick, blue] (0,0) -- (4,0) -- (3,2.5) -- (0,2) -- cycle;

		% Vértices
		\fill[black] (0,0) circle (2pt) node[anchor=north] {Vértice 1};
		\fill[black] (4,0) circle (2pt) node[anchor=north west] {Vértice 2};
		\fill[black] (3,2.5) circle (2pt) node[anchor=south] {Vértice 3};
		\fill[black] (0,2) circle (2pt) node[anchor=south east] {Vértice 4};

		%Título
		%\node[below] at (2,-1) {\textbf{Figura 1: Interseção de hiperplanos em $\mathbb{R}^2$ (polígono)}};

	\end{tikzpicture}
	\caption{Interseção de hiperplanos em \(\mathbb{R}^2\)}
	\end{figure}

	%\vspace{2cm}
	\item No espaço tridimensional ($\mathbb{R}^3$): Um vértice de um poliedro é o ponto de encontro de três faces, que correspondem a hiperplanos em um espaço tridimensional.

	% Figura no espaço tridimensional (R^3)
	\begin{figure}[H]
	\centering
	\begin{tikzpicture}[scale=0.8]
		% Poliedro (um tetraedro simples)
		\fill[blue!10,opacity=0.7] (0,0,0) -- (4,0,0) -- (2,3,0) -- cycle; % Base
		\fill[blue!20,opacity=0.7] (0,0,0) -- (4,0,0) -- (2,1.5,3) -- cycle; % Face lateral 1
		\fill[blue!30,opacity=0.7] (0,0,0) -- (2,3,0) -- (2,1.5,3) -- cycle; % Face lateral 2
		\fill[blue!40,opacity=0.7] (4,0,0) -- (2,3,0) -- (2,1.5,3) -- cycle; % Face lateral 3

		% Arestas do poliedro
		\draw[thick, black] (0,0,0) -- (4,0,0);
		\draw[thick, black] (4,0,0) -- (2,3,0);
		\draw[thick, black] (2,3,0) -- (0,0,0);
		\draw[thick, black] (0,0,0) -- (2,1.5,3);
		\draw[thick, black] (4,0,0) -- (2,1.5,3);
		\draw[thick, black] (2,3,0) -- (2,1.5,3);

		% Vértices
		\fill[black] (0,0,0) circle (2pt) node[anchor=north east] {Vértice 1};
		\fill[black] (4,0,0) circle (2pt) node[anchor=north] {Vértice 2};
		\fill[black] (2,3,0) circle (2pt) node[anchor=south] {Vértice 3};
		\fill[black] (2,1.5,3) circle (2pt) node[anchor=south east] {Vértice 4};

		% Título
		%\node[below] at (2,-1,0) {\textbf{Figura 2: Interseção de hiperplanos em $\mathbb{R}^3$ (poliedro)}};
	\end{tikzpicture}
	\caption{Interseção de Hiperplanos em $\mathbb{R}^3$}
	\end{figure}
\end{itemize}

Portanto, o conceito algébrico de ponto extremo está diretamente relacionado ao conceito geométrico de vértice. Concluímos que os pontos extremos de um conjunto viável de um PPL são, geometricamente, os vértices do poliedro formado por esse conjunto.


Pontos extremos desempenham um papel importante na Programação Linear, pois como veremos adiante, a solução ótima de um PPL sempre está em um ponto extremo do conjunto viável. Para o método simplex, a existência de pontos extremos degenerados requer precauções, pois esses podem afetar o desempenho do método, inclusive fazendo rodar indefinitivamente.

\subsection{Raios e Direções}

Um raio também é uma classe de conjuntos convexos, que são definidos a seguir.

\begin{def:raio}
	Um raio $r$ é uma coleção de pontos dados na forma
	\begin{equation*}
		r = \{\mathbf{x} \in \mathbb{R}^n \ |\  \mathbf{x} + \lambda \mathbf{d}, \lambda > 0\}
	\end{equation*}
	onde $\mathbf{d}$ é um vetor em $\mathbb{R}^n$ não nulo e $\lambda$ é um real positivo
\end{def:raio}

Podemos também interpretar raios geometricamente como sendo uma semirreta com origem em $\mathbf{x}$, dito como o vértice do raio, e que se estende na direção do vetor $\mathbf{d}$, chamado de diretor ou direção do raio.

\begin{def:direção}
	Seja $X$ um conjunto convexo contido no $\mathbb{R}^n$. O vetor não nulo $\vec{d} \in \mathbb{R}^n$ é uma direção de $X$ se para qualquer ponto $\vec{x_0} \in X$ e qualquer $\lambda \geq 0$ é verdade que $\vec{x_0} + \lambda \vec{d} \in X$
\end{def:direção}

A Definição expande a noção de direção para qualquer conjunto convexo além dos raios. Podemos dizer, então, que um conjunto $X$ possui uma direção se ele contém um raio. Observe também que um conjunto convexo $X$ possui uma direção se, e somente se, ele não é limitado. A demonstração disso fica como exercício para o leitor\footnote{Tente argumentar o porquê de $X$ possuir uma direção implicar em ele não poder ter uma bola o contendo. Para a recíproca, argumente o porquê de um conjunto ser contido por uma bola não poder ter uma direção na qual ele se estende indefinitivamente.}.

Agora iremos nos aprofundar um pouco mais nas características algébricas de uma direção para um conjunto poliédrico $X = \{\vec{x}\ |\ A\vec{x} = \vec{b}, \vec{x} \geq 0\}$. Se temos que para todo $\vec{x} \in X$, $\vec{x} + \lambda \vec{d} \in X$, então é verdade que
\begin{gather*}
	A (\vec{x} + \lambda \vec{d}) \leq \vec{b} \\
	\vec{x} + \lambda \vec{d} \geq 0
\end{gather*}

Porém, para que $\vec{d}$ seja uma direção de $X$, temos que ele deve satisfazer algumas condições algébricas a mais, que serão formalizadas no seguinte enunciado.

\begin{prop:direção}
	Seja $X = \{\vec{x}\ |\ A\vec{x} = \vec{b}, \vec{x} \geq 0\}$ um conjunto poliédrico. O vetor $\vec{d}$ é uma diereção de $X$ se, e somente se,
	\begin{equation*}
		\vec{d} > 0, \quad A\vec{d} \leq 0
	\end{equation*}

	\begin{proof}
		Com efeito, se $\vec{d}$ é uma direção de $X$, então
		\begin{align*}
			A (\vec{x} + \lambda \vec{d}) &\leq \vec{b} \\
			A \vec{x} + A (\lambda \vec{d}) & \leq b
		\end{align*}
		mas como $A \vec{x} \leq \vec{b}$, então é preciso que
		\begin{equation*}
			A\vec{d} \leq 0
		\end{equation*}
		Ademais, se $\vec{x} + \lambda \vec{d} \geq 0$, então
		\begin{align}
			\vec{d} \geq \frac{1}{\lambda} -\vec{x}
		\end{align}
		como $\lambda \geq 0$ pode ser arbitrariamente grande, então
		$\lim_{\lambda \to \infty} -\vec{x} = 0$, e com isso
		\[d \geq 0\].

		Por outro lado, se $\vec{d} > 0$ e $A \vec{d} \leq 0$, então para $\lambda \geq 0$ temos que
		\[A (\lambda \vec{d}) \leq 0\]
		e se $A \vec{x} \leq b$, então
		\begin{align*}
			A \vec{x} + A (\lambda \vec{d}) &\leq \vec{b} \\
			A (\vec{x} + \lambda \vec{d}) &\leq \vec{b}
		\end{align*}
		e como $\vec{x} \in X$ é arbitrário, então $d$ é uma direção de $X$.
	\end{proof}
\end{prop:direção}

\begin{def:direção extrema}
	Seja $X$ um conjunto convexo e $\vec{d}$ uma direção desse conjunto. O vetor $\vec{d}$ é chamado de direção extrema se ele não pode ser dado como combinação linear positiva de outras direções desse conjunto.
\end{def:direção extrema}

A ideia de direções extremas é análoga a de pontos extremos. O vetor $\vec{d}$ é direção extrema se não existe outras duas direções $\vec{d}_1$ e $\vec{d}_2$ tal que
\begin{gather*}
	\lambda_1\vec{d}_1 + \lambda_2\vec{d}_2 = \vec{d} \\
	\lambda_1, \lambda_2 \geq 0
\end{gather*}
Os raios cuja a direção é dada por uma direção extrema são chamados de raios extremos.

Os conceitos de raios e direções podem ser usados para definir os cones convexos. Seja $C \subset \mathbb{R}^n$ um cone definido como \[C = \{\vec{x}\ |\ \lambda\vec{x}, \lambda \geq 0\}\]Podemos perceber que o subconjunto de $C$ dado pelos múltiplos escalares positivos de $\vec{x} \in C$ formam um raio com vértice na origem do $\mathbb{R}^n$ com direção dada por $\vec{x}$. Dessa forma, um cone pode ser definido a partir de suas direções, mas nem todas são necessárias para tal, já que podemos usar somente as suas direções extremas, que formam um conjunto minimal gerador das demais.

Portanto, se $C$ é um cone e $D = \{\vec{d}_1, \vec{d}_2, \ldots, \vec{d}_n\}$ o conjunto das suas direções extremas, então o cone pode ser dado como
\begin{equation*}
	C = \{\vec{x}\ |\ \vec{x} = \displaystyle\sum_{j = 1}^{n}
			\lambda_j \vec{d}_j, \lambda_j \geq 0\}
\end{equation*}
Porém iremos formalizar isso num resultado futuro.
\begin{figure}[H]
\centering
\begin{tikzpicture}[scale=2]

	% Eixos coordenados
	\draw[->] (-1.5, 0) -- (1.5, 0) node[below right] {\(x_1\)};
	\draw[->] (0, -1.5) -- (0, 1.5) node[above left] {\(x_2\)};

	% Semirretas (vetores que formam o cone)
	\draw[thick, red, ->] (0, 0) -- (1, 0.5) node[midway, above right] {};
	\draw[thick, red, ->] (0, 0) -- (0.5, 1) node[midway, above right] {};

	% Anotação para a direção extrema
	\draw[->, thick] (1.2, 0.2) -- (0.6, 0.3) node[pos=-0.1, right] {\textbf{Direção extrema}};


	% Região do cone (preenchida)
	\fill[red, opacity=0.3] (0, 0) -- (1, 0.5) -- (0.5, 1) -- cycle;

	% Vetores dentro do cone (exemplo de combinações convexas)
	%\draw[thick, blue, ->] (0, 0) -- (0.8, 0.6) node[midway, above] {};
	%\draw[thick, blue, ->] (0, 0) -- (0.4, 0.4) node[midway, above] {};

	% Adicionando um label para o cone
	\node[red] at (1, 1.2) {\textbf{Cone}};
\end{tikzpicture}
\caption{Representação de um Cone e seus raios extremos em \(\mathbb{R}^2\)}

\label{fig:cone}
\end{figure}

\subsection{Faces e Arestas}

Nesta seção apenas abordaremos algumas definições algébricas para conceitos historicamente geométricos. A primeira delas será a \textbf{face}.

\begin{def:face}
	Seja $X$ um conjunto convexo e $F = \{\vec{x} \in X\ |\ B\vec{x} = \vec{c}\}$ em que $B\vec{x} = \vec{c}$ é o sistema linear cujas equações é um subconjunto não vazio dos hiperplanos definidores de $X$. O conjunto $F$ é então chamado de face de $X$.
\end{def:face}

Em outras palavras uma face é um subconjunto de $X$ cujos elementos são solução de um sistema linear dado por 1 ou mais hiperplanos de $X$. Em dimensão três, quando pensamos em faces de um poliedro, pensamos naquelas que tem dimensão dois, mas a nossa definição permite que outros elementos de dimensão menores, como as arestas e os vértices desse poliedro também sejam chamados de ``faces''.

Na verdade, em dimensão $n$ podemos ter faces de qualquer dimensão entre 1 e $n$. As faces de dimensão $n - 1$, como as faces propriamente ditas de um poliedro em dimensão 3, ou as arestas de um polígono bidimensional, são chamadas de \textbf{facetas}. Facetas também podem ser entendidas como a parte viável (que está em $X$) de um hiperplano definidor.  As faces com dimensão entre $n - 1$ e $0$ são chamadas de faces próprias de $X$, enquanto aquela de dimensão $n$, que é o próprio $X$ é chamada de face imprópria\footnote{Se nossa definição considerasse o conjunto vazio também como face de um conjunto convexo, então ele também seria uma face imprópria}.

Vamos agora dizer que $r(F)$ da face $F$ de um conjunto convexo $X$ é o número mínimo de hiperplanos necessários para definir $F$ como o conjunto solução de um sistema linear. Esse número, para pontos extremos, que são faces de dimensão 0, é $n$, como vimos no teorema \ref{thm:ponto extremo}. Para uma aresta, que é um segmento de reta, e possui, pois, dimensão 1, esse número é $n - 1$. Podemos concluir que no geral a dimensão $\dim(F)$ de uma face $F$ e o número $r(F)$ estão relacionadas por
\begin{equation*}
	r(F)= n - \dim(F)
\end{equation*}

Ainda sobre arestas, sabemos da geometria clássica que elas sempre conectam dois vértices, par esse chamado de adjacentes. Como vimos, pontos extremos equivalem aos vértices de um poliedro convexo $n$ dimensional, e, analogamente, os pontos extremos que são conectados por uma aresta (o que nem sempre ocorre, visto que os poliedros podem ser ilimitados) são também chamados de \textbf{adjacentes}. O método simplex, ao caminhar pelos vértices da região viável em busca da solução ótima do Programa Linear, sempre o faz caminhando pelas arestas da região viável, indo de um vértice para outro que seja adjacente.

