%==================== Definições =======================
\newtheorem{def:ponto extremo}[def:conjunto convexo]{Definição}

\newtheorem{def:raio}[def:conjunto convexo]{Definição}

\newtheorem{def:direção}[def:conjunto convexo]{Definição}

\newtheorem{def:direção extrema}[def:conjunto convexo]{Definição}

\newtheorem{def:ponto degenerado}[def:conjunto convexo]{Definição}


%================== Proposições ================
\newtheorem{prop:combinação convexa}{Proposição}[chapter]

\newtheorem{prop:hiperplano e ponto extremo}[prop:combinação convexa]{Proposição}

%=================== Teoremas ===================

\newtheorem{thm:ponto extremo}{Teorema}[chapter]

\section{Geometria dos Espaços Convexos}

\subsection{Pontos Extremos}

O conceito de ponto extremo de um conjunto convexo $\mathbb{V}$ é uma especialização do conceito de ``independência linear'' para combinações lineares convexas.

\begin{def:ponto extremo}
	Um vetor $x$ num conjunto convexo $\mathbb{V}$ é chamado ponto extremo de $\mathbb{V}$ se ele não pode ser expresso como combinação linear convexa de nenhum outro subconjunto de $\mathbb{V}$. 
\end{def:ponto extremo}

De grosso modo, podemos dizer que os pontos extremos de um conjunto convexo é um conjunto linearmente independente se considerarmos apenas suas combinações convexas. Além disso, é possível ver também que os pontos extremos formam uma ``base'' para conjuntos convexos, ou seja, qualquer ponto de um conjunto convexo pode ser dado como combinação convexa dos pontos extremos, mas não se preocupe com isso agora, pois iremos explorar essa ideia mais adiante.

Agora, iremos explorar mais o significado geométrico de um ponto extremo. Mas para isso, iremos demonstrar um resultado que irá nos auxiliar nesta tarefa.

\begin{prop:combinação convexa}
	\label{prop:combinação convexa}
	Se $\alpha$ e $\beta$ são números reais e $\gamma$ é uma combinação linear estritamente convexa desses números, tal que
	\[\lambda \alpha + (1- \lambda) \beta = \gamma,\ 0 < \lambda < 1\]e que $\alpha \leq \gamma$ e $\beta \leq \gamma$, então \[\alpha = \beta = \gamma\]
	
	\begin{proof}
		Primeiros vamos explicitar a ideia geométrica de uma combinação linear convexa para números reais. Tradicionalmente, o conjunto dos reais é compreendido como uma reta cujos pontos estão, respectivamente, associados a um número real. Dessa forma, o conjunto das combinações lineares convexas entre dois pontos da reta real é o segmento de reta que une esses dois pontos.
		
		Dito isso, se $\gamma$ é uma combinação convexa de $\alpha$ e $\beta$, então é verdade que \[\alpha \leq \gamma \leq \beta\]uma vez que o ponto associado a $\gamma$ deve estar disposto no segmento unindo os pontos associados a $\alpha$ e $\beta$.	Visto que, por hipótese
		\begin{align*}
			\alpha \leq \gamma \quad \beta \leq \gamma
		\end{align*}
		então, da expressão à direita segue que $\beta = \gamma$, ao passo que
		\begin{gather*}
			\lambda \alpha + (1 - \lambda) \gamma = \gamma \\
			\lambda \alpha = \lambda \gamma \\
			\alpha = \gamma
		\end{gather*}
	\end{proof}
\end{prop:combinação convexa}

Essa proposição, num primeiro momento, parece estar desconectada da ideia dos pontos extremos, mas apesar de simples, ela é fundamental para construirmos a geometria daquele conceito. Vamos enunciar mais uma proposição para seguirmos com o nosso objetivo. Em resumo, ela diz que, se um ponto em um hiperplano é combinação convexa de outros dois que compartilham o mesmo semiespaço, então esses dois últimos também estão no hiperplano em questão.

\begin{prop:hiperplano e ponto extremo}
	\label{prop:hiperplano e ponto extremo}
	Sejam os vetores $\vec{x}'$ e $\vec{x}''$ pertencentes a um mesmo semiespaço $H^-$. Se uma combinação estritamente convexa desses vetores está no hiperplano $H$, então ambos os vetores pertencem a esse hiperplano.
	
	\begin{proof}
		Digamos que $H^-$ seja definido como
		\[H^- = \{\vec{x}\ |\ \transp{a}\vec{x} \leq b\}\]
		Com efeito, se $\vec{x}'$ e $\vec{x}''$ estão em $H^-$, então
		\[\transp{a}\vec{x}' \leq b \quad \transp{a}\vec{x}'' \leq b\]
		Seja agora $\bar{\vec{x}}$ uma combinação convexa de $\vec{x}'$ e $\vec{x}''$ tal que 
		\[\lambda \vec{x}' + (1 - \lambda) \vec{x}'' = \bar{\vec{x}}\]
		com $0 < \lambda < 1$ e $\bar{\vec{x}} \in H$, isto é
		\[\transp{a}\bar{\vec{x}} = b\]
		Disso segue que
		\[\lambda (\transp{a}\vec{x}') + (1 - \lambda) (\transp{a}\vec{x}'') = b\]
		
		Como bem sabemos, o resultado do produto escalar entre vetores é um número real, e o que a expressão acima está nos mostrando é que $b$ é um combinação convexa dos números reais $\transp{a}\vec{x}'$ e $\transp{a}\vec{x}''$. Ora, mas também temos que $\vec{x}', \vec{x}'' \in H^-$, o que implica pela proposição \ref*{prop:combinação convexa} que
		\[\transp{a}\vec{x}' = b \quad \transp{a}\vec{x}'' = b\]ou seja, $\vec{x}', \vec{x}'' \in H$
	\end{proof} 
\end{prop:hiperplano e ponto extremo}

Com esse último resultado em mãos, agora iremos caracterizar os pontos extremos a partir das noções já vistas de semiespaços e hiperplanos. Para viés de síntese, sendo $X = \{x\ |\ A\vec{x} = \vec{b},\ \vec{x} \geq 0\}$, com $A$ sendo $m \times n$, iremos chamar os $n + m$ hiperplanos associados aos semiespaços definidores de $X$ (as $n$ restrições agrupadas em $A$ mais a $m$ restrições de não negatividade) de hiperplanos definidores de $X$.  

\begin{thm:ponto extremo}
	Um vetor $\bar{\vec{x}}$ é ponto extremo de um conjunto convexo $X = \{x\ |\ A\vec{x} = \vec{b},\ \vec{x} \geq 0\}$ se $\bar{\vec{x}}$ pertence a pelo menos $n$ hiperplanos linearmente independentes que definem $X$.
	
	\begin{proof}
		Suponha por contradição que $\bar{\vec{x}}$ pertence a pelo menos $n$ hiperplanos linearmente independentes definidores de $X$, mas não é um ponto extremo. Com efeito, $\bar{\vec{x}}$ pode ser dado como combinação estritamente convexa de vetores $\vec{x}'$ e $\vec{x}''$ em $X$. Pela Proposição \ref{prop:hiperplano e ponto extremo} que tanto $\bar{\vec{x}}'$ quanto $\bar{\vec{x}}''$ pertencem a esses $n$ hiperplanos definidores.
		
		Entretanto, o sistema linear dado pelas equações desses hiperplanos é uma sistema quadrado $n \times n$ e com solução única, uma vez que eles são linearmente independentes. Portanto \[\vec{x}' = \vec{x}''\]o que contradiz a hipótese, logo $\bar{\vec{x}}$ não pode ser uma combinação estritamente convexa daqueles vetores.
		
		Por outro lado, suponhamos agora que $\bar{\vec{x}}$ pertença a $r < n$ hiperplanos linearmente independentes, e que o sistema linear dado por eles é \[B\vec{x} = \vec{c}\], onde $B$ é uma matriz $r \times n$. O que faremos agora será construir vetores $\vec{x}'$ e $\vec{x}''$ dos quais $\bar{\vec{x}}$ é uma combinação estritamente convexa.  
		
		Por ser uma matriz retangular larga (mais colunas do que linhas), o posto de $B$ é no máximo $r$, logo ela não pode ter posto cheio, o que implica que existe um vetor $\vec{d} \neq 0$ tal que $B \vec{d} = 0$. Digamos agora que
		\begin{equation*}
			\vec{x}' = B(\bar{\vec{x}} + \epsilon \vec{d})
			\quad
			\vec{x}'' = B(\bar{\vec{x}} - \epsilon \vec{d})
		\end{equation*}
		Por conseguinte temos que tanto $\vec{x}'$ quanto $\vec{x}''$ satisfazem o sistema $B\vec{x} = \vec{c}$, pertencendo aos $r$ hiperplanos nos quais $\bar{\vec{x}}$ também está. 
		
		Para que $\vec{x}'$ e $\vec{x}''$ satisfaçam as demais $n - r$ restrições, basta tomarmos um $\epsilon > 0$ suficientemente pequeno. Isso porque, se pensarmos geometricamente,  $\vec{x}'$ e $\vec{x}''$ estão ambos nas semirretas com vértice em $\bar{\vec{x}}$ e cujas direções são dadas por $\vec{d}$ e $-\vec{d}$ respectivamente. Desse modo, para que não avancemos para além da fronteira de $X$, basta que $\epsilon$ seja pequeno o suficiente para tal. O fato de termos construído o vetor $\vec{d}$ vem da ideia de que se temos um sistema com infinitas soluções e conhecemos uma solução dele, podemos encontrar outras somando essa solução com uma vinda do sistema homogêneo.
		
		Por fim, podemos tomar simplesmente $\bar{\vec{x}} = 0.5 \vec{x}' + 0.5 \vec{x}$ e concluímos que, se $\bar{\vec{x}}$ não pertence ao menos $n$ dos hiperplanos que definem $X$, então $\bar{\vec{x}}$ não pode ser um ponto extremo. 
	\end{proof}
\end{thm:ponto extremo}

Você talvez deve estar se perguntando agora sobre os pontos extremos que pertencem a mais do que $n$ dos hiperplanos definidores. Para esses pontos usamos uma nomenclatura especial, que iremos formalizar na definição a seguir.

\begin{def:ponto degenerado}
	Seja $X = \{\vec{x}\ |\ A\vec{x} = \vec{b}, \vec{x} \geq 0\}$ um conjunto convexo em que $A$ é $m \times n$. Se $\vec{x}$ é um ponto extremo de $X$ tal que $\vec{x}$ pertença a mais do que $n$ dos $n + m$ hiperplanos definidores de $X$, então $\vec{x}$ é chamado de \textbf{ponto extremo degenerado}. O número excedente de hiperplanos que possuem $\vec{x}$ é chamado de \textbf{ordem de degeneração}.
\end{def:ponto degenerado}

Em resumo, podemos definir um ponto extremo com $n$ hiperplanos definidores linearmente independentes do nosso conjunto convexo $X$. Se existe mais de uma maneira de definir esse ponto extremos com $n$ hiperplanos, então esse ponto é um ponto degenerado.

%Pontos extremos são então soluções de um sistema dado por um subconjunto linearmente independente das restrições na forma padrão de um PPL. Sabemos da Álgebra Linear que em $\mathbb{R}^n$, a interseção de $n$ hiperplanos LI é um ponto, esse que também pode ser chamado de vértice. Pense bem, no plano, os hiperplanos são retas, e quando desenhamos um polígono, os vértices dele são as interseções entre duas retas. Da mesma forma, em poliedros em três dimensões, um vértice é o encontro de 3 faces, que são os hiperplanos de um espaço de dimensão 3. Portanto, o conceito algébrico de ponto extremos está diretamente atrelado ao conceito de vértice da geometria, e podemos dizer que os pontos extremos são então os vértices do poliedro dado pelo conjunto viável do PPL.

Pontos extremos de um PPL (Problema de Programação Linear) são soluções de um sistema formado por um subconjunto linearmente independente de restrições, na forma padrão. Da Álgebra Linear, sabemos que, em $\mathbb{R}^n$, a interseção de $n$ hiperplanos linearmente independentes é um ponto, o qual também pode ser chamado de \textbf{vértice}.

Para visualizar isso, considere exemplos em dimensões menores:
\begin{itemize}
	\item No plano ($\mathbb{R}^2$): Os hiperplanos são representados por retas. Ao desenharmos um polígono, os vértices do polígono correspondem às interseções de pares de retas.
	% Figura no plano (R^2)
	\begin{figure}[H]
	\centering
	\begin{tikzpicture}
		% Desenho do polígono no plano
		\fill[blue!10] (0,0) -- (4,0) -- (3,2.5) -- (0,2) -- cycle;
		
		% Linhas (hiperplanos no plano)
		%\draw[thick, red] (0,2) -- (4,1.5) node[anchor=south east] {Hiperplano 1};
		%\draw[thick, green!70!black] (0,0) -- (4,3) node[anchor=south] {Hiperplano 2};
		%\draw[thick, purple] (0,3.5) -- (4,-0.5) node[anchor=north] {Hiperplano 3};
		
		% Polígono (interseção das restrições)
		\draw[thick, blue] (0,0) -- (4,0) -- (3,2.5) -- (0,2) -- cycle;
		
		% Vértices
		\fill[black] (0,0) circle (2pt) node[anchor=north] {Vértice 1};
		\fill[black] (4,0) circle (2pt) node[anchor=north west] {Vértice 2};
		\fill[black] (3,2.5) circle (2pt) node[anchor=south] {Vértice 3};
		\fill[black] (0,2) circle (2pt) node[anchor=south east] {Vértice 4};
		
		%Título
		%\node[below] at (2,-1) {\textbf{Figura 1: Interseção de hiperplanos em $\mathbb{R}^2$ (polígono)}};

	\end{tikzpicture}
	\caption{Interseção de hiperplanos em \(\mathbb{R}^2\)}
	\end{figure}
	
	%\vspace{2cm}
	\item No espaço tridimensional ($\mathbb{R}^3$): Um vértice de um poliedro é o ponto de encontro de três faces, que correspondem a hiperplanos em um espaço tridimensional.
	
	% Figura no espaço tridimensional (R^3)
	\begin{figure}[H]
	\centering
	\begin{tikzpicture}[scale=0.8]
		% Poliedro (um tetraedro simples)
		\fill[blue!10,opacity=0.7] (0,0,0) -- (4,0,0) -- (2,3,0) -- cycle; % Base
		\fill[blue!20,opacity=0.7] (0,0,0) -- (4,0,0) -- (2,1.5,3) -- cycle; % Face lateral 1
		\fill[blue!30,opacity=0.7] (0,0,0) -- (2,3,0) -- (2,1.5,3) -- cycle; % Face lateral 2
		\fill[blue!40,opacity=0.7] (4,0,0) -- (2,3,0) -- (2,1.5,3) -- cycle; % Face lateral 3
		
		% Arestas do poliedro
		\draw[thick, black] (0,0,0) -- (4,0,0);
		\draw[thick, black] (4,0,0) -- (2,3,0);
		\draw[thick, black] (2,3,0) -- (0,0,0);
		\draw[thick, black] (0,0,0) -- (2,1.5,3);
		\draw[thick, black] (4,0,0) -- (2,1.5,3);
		\draw[thick, black] (2,3,0) -- (2,1.5,3);
		
		% Vértices
		\fill[black] (0,0,0) circle (2pt) node[anchor=north] {Vértice 1};
		\fill[black] (4,0,0) circle (2pt) node[anchor=north] {Vértice 2};
		\fill[black] (2,3,0) circle (2pt) node[anchor=south] {Vértice 3};
		\fill[black] (2,1.5,3) circle (2pt) node[anchor=south] {Vértice 4};
		
		% Título
		%\node[below] at (2,-1,0) {\textbf{Figura 2: Interseção de hiperplanos em $\mathbb{R}^3$ (poliedro)}};
	\end{tikzpicture}
	\caption{Interseção de Hiperplanos em $\mathbb{R}^3$}
	\end{figure}
\end{itemize}

Portanto, o conceito algébrico de ponto extremo está diretamente relacionado ao conceito geométrico de vértice. Concluímos que os pontos extremos de um conjunto viável de um PPL são, geometricamente, os vértices do poliedro formado por esse conjunto.
     

Pontos extremos desempenham um papel importante na Programação Linear, pois como veremos adiante, a solução ótima de um PPL sempre está em um ponto extremo do conjunto viável. Para o método simplex, a existência de pontos extremos degenerados requer precauções, pois esses podem afetar o desempenho do método, inclusive fazendo rodar indefinitivamente.   

\subsection{Raios e Direções}

Um raio também é uma classe de conjuntos convexos, que são definidos a seguir.

\begin{def:raio}
	Um raio $r$ é uma coleção de pontos dados na forma
	\begin{equation*}
		r = \{\mathbf{x} \in \mathbb{R}^n \ |\  \mathbf{x} + \lambda \mathbf{d}, \lambda > 0\}
	\end{equation*}
	onde $\mathbf{d}$ é um vetor em $\mathbb{R}$ não nulo e $\lambda$ é um real positivo
\end{def:raio}

Podemos também interpretar raios geometricamente como sendo uma semirreta com origem em $\mathbf{x}$, sendo chamado de vértice do raio, e que se estende na direção do vetor $\mathbf{d}$, chamado de diretor ou direção do raio.

\begin{def:direção}
	Seja $\mathbb{V}$ um conjunto convexo contido no $\mathbb{R}^n$. O vetor não nulo $\vec{d} \in \mathbb{R}^n$ é uma direção de $X$ com vértice em $\vec{x_0} \in X$ se para qualquer $\lambda \geq 0$ é verdade que $\vec{x_0} + \lambda \vec{d} \in X$  
\end{def:direção}

A definição expande a noção de direção para qualquer conjunto convexo além dos raios. Dessa forma, uma direção pode ser entendida como uma semirreta que está contida num conjunto convexo. Disso segue que um conjunto convexo que possui uma direção é sempre ilimitado.

\begin{def:direção extrema}
	Seja $X$ um conjunto convexo e $\vec{d}$ uma direção desse conjunto. O vetor $\vec{d}$ é chamado de direção extrema se ele não pode ser dado como combinação linear positiva de outras direções desse conjunto.
\end{def:direção extrema}

A ideia de direções extremas é análoga a de pontos extremos. O vetor $\vec{d}$ é direção extrema se não existe outras duas direções $\vec{d}_1$ e $\vec{d}_2$ tal que
\begin{gather*}
	\lambda_1\vec{d}_1 + \lambda_2\vec{d}_2 = \vec{d} \\
	\lambda_1, \lambda_2 \geq 0
\end{gather*}
Os raios cuja a direção é dada por uma direção extrema são chamados de raios extremos.

Os conceitos de raios e direções podem ser usados para definir os cones convexos. Seja $C$ um cone definido como \[C = \{\vec{x}\ |\ \lambda\vec{x}, \lambda \geq 0\}\]Podemos perceber que o subconjunto de $C$ dado pelos múltiplos escalares positivos de $\vec{x} \in C$ formam um raio que emana da origem com direção dada por $x$. Dessa forma, um cone é pode ser definido a partir de suas direções, mas nem todas são necessárias para tal, já que podemos usar somente as suas direções extremas, que formam um conjunto minimal.

Portanto, se $C$ é um cone e $D = \{\vec{d}_1, \vec{d}_2, \ldots, \vec{d}_n\}$ o conjunto das suas direções extremas, então o cone pode ser dado como
\begin{equation*}
	C = \{\vec{x}\ |\ \vec{x} = \displaystyle\sum_{j = 1}^{n}
			\lambda_j \vec{d}_j, \lambda_j \geq 0\}
\end{equation*} 
