\section{O Problema de Programação Linear}

Como dito anteriormente, o problema geral da programação matemática é otimizar uma função - seja maximizá-la ou minimizá-la - dada um conjunto de restrições sobre suas variáveis. Quando tanto a função quanto suas restrições são lineares, chamamos o problema de programação matemática de \textbf{Problema de Programação Linear} (PPL). 

Dada uma função $f: \mathbb{R}^n \to \mathbb{R}$ definida como \[f(x) = c^tx,\ c \in \mathbb{R}^n\]e um conjunto de inequações
\begin{gather*}
	a_{11}x_1 + a_{12}x_2 + \ldots + a_{1n}x_n \leq b_1 \\
	a_{21}x_1 + a_{22}x_2 + \ldots + a_{2n}x_n \leq b_2 \\
	\vdots \\
	a_{m1}x_1 + a_{m2}x_2 + \ldots + a_{mn}x_n \leq b_m
\end{gather*} chamamos a função a ser otimizada $f(x)$ de \textbf{função objetivo}, as variáveis $x_1, \ldots, x_2$ de \textbf{variáveis de decisão} e o conjunto de inequações de \textbf{restrições}. A verdade que as restrições não necessariamente precisam ser da forma $a^t_i x \leq b_i$, podendo ser da forma $a^t_i x \geq b_i$ ou  $a^t_i x = b_i$. O conjunto dos vetores $x \in \mathbb{R}$ que são solução para o Problema de Programação Linear é chamado de \textbf{conjunto viável}, ou factível.

\subsection{Notação Matricial}

Para facilitar a notação, podemos usar a representação matricial para o conjunto de restrições. Se $a_{ij}$ é a entrada da i-ésima linha e j-ésima coluna da matriz $A$, de modo que
\begin{gather*}
A = \begin{bmatrix}
	a_{11} & a_{12} & \ldots & a_{1n} \\
	a_{21} & a_{22} & \ldots & a_{2n} \\
	\vdots & \vdots & \ddots & \vdots \\
	a_{m1} & a_{m2} & \ldots & a_{mn} \\
\end{bmatrix}
\end{gather*}
então reduzimos as restrições simplesmente a \[Ax \leq b\], tal que
\begin{align*}
x = \begin{bmatrix}
	x_1 \\ x_2 \\ \vdots \\ x_n
\end{bmatrix}
\hspace{1cm}
b = \begin{bmatrix}
	b_1 \\ b_2 \\ \vdots \\ b_m
\end{bmatrix}
\end{align*}
Notação de desigualdades para matrizes pode parecer estranha, e realmente é. Mas ela é muito comum ao descrever problemas como os quais estamos lidando. Nesta notação, temos que: se $a_{i}$ é o vetor cujas componentes são as entradas da i-ésima linha de $A$, então a desigualdade acima diz que, para todo $i=1,2,\ldots,m$ é verdade que \[a^t_ix \leq b_i\].

\subsection{Restrições de Não Negatividade}
Por se tratar de um ramo que se desenvolveu em boa parte a partir de problemas concretos, a programação linear lida com variáveis de decisão que geralmente representam quantidades de alguma grandeza do mundo real, e por isso são não negativas. Logo, é comum a adição ao conjunto de restrições a expressão \[x \geq 0\]que significa que todas as componentes do vetor das variáveis de decisão são não negativas.

Mas isso não anula a existência de variáveis que possam assumir valores negativos, que geralmente são chamadas de \textbf{variáveis irrestritas}. Contudo, se $x_j$ for uma variável irrestrita, então podemos expressá-la como $x_j = x_j^+ - x_j^-$, tal que $x_j^+ \geq 0$ e $x_j^- \geq 0$. Portanto, podemos expressar uma variável irrestrita a partir de outras duas não negativas. Ademais, se temos $k$ variáveis irrestritas, necessitaremos de $k$ variáveis $x^+$ e apenas uma $x^-$, que será a mais negativa das expressões $x_j = x^+_j - x^-_j$. 

Se $x_j > l_j$ então podemos trocá-la por $x'_j = x_j - l_j$, que será não negativa. De forma análoga, se tivermos que $x_j < u_j$, então podemos trocar essa variável por $x'_j = u_j - x_j$, que também será não negativa.

\subsection{Inequações}

Podemos transitar facilmente entre os tipos de inequações apenas multiplicando-as por $-1$. Por exemplo, uma inequação da forma ``menor ou igual'' como $2x_1 + 3x_2 \leq 5$  é equivalente a $-2x_1 - 3x_2 \geq -5$, que é da forma ``maior ou igual''.

Também é importante sabermos com transitar entre equações e inequações, pois o método simplex, por exemplo, aceita a entrada da restrições apenas como igualdades. Desse modo, se toemos uma inequação como $4x_1 + x_2 \geq 10$, para a transformar numa igualdade, iremos subtrair do primeiro membro uma nova variável não negativa $x_3$, que é chamada de \textbf{variável de folga}, obtendo assim a igualdade $4x_1 + x_2 - x_3 = 10$. Da mesma forma, se tivéssemos a inequação $7x_1 + 2x_2 \leq 9$, poderíamos somar a variável de folga $x_3$ a inequação para obter uma igualdade: $7x_1 + 2x_2 + x_3 = 9$.

\subsection{Problemas de Maximização e Minimização}

A otimização da função objetivo $f(x) = z$ pode se referir a sua maximização, ou sua minimização. Contudo, ambos os problemas são equivalentes. Se o problema é $\max z = c^tx$, temos que ele é equivalente ao $ \min z' = -c^tx $, tal que se $\hat{z}$ é o valor ótimo do primeiro e $\hat{z}'$ o do segundo, então $\hat{z} = -\hat{z}'$. 

Logo, para transformar passar de um problema para outro, basta multiplicarmos os coeficientes da função por $-1$.

\begin{equation*}
	\text{maximizar\ } c^tx = \text{minimizar\ } -c^tx
\end{equation*}

\subsection{Forma Padrão e Canônica}
Um PPL está na forma \textbf{padrão} se todas as restrições são igualdades e todas as variáveis são não negativas. O método simplex, por exemplo, resolve programas lineares apenas quando estão na forma padrão.

A forma canônica de um PPL depende do tipo de otimização que estamos buscando. Se o problema é de maximização, então sua forma canônica é dada por
\begin{align*}
\text{Maximizar:} \quad z & = c^t x \\
\text{sujeito a:} \quad Ax & \leq b \\
				  \quad x & \geq 0
\end{align*}
e quando de minimização, então sua forma canônica é
\begin{align*}
	\text{Minimizar:} \quad z & = c^t x \\
	\text{sujeito a:} \quad Ax  & \geq b \\
	\quad  x  & \geq 0
\end{align*}
