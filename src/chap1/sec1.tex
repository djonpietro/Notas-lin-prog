\section{Um Breve Histórico da Programação Linear}

A história dos métodos de otimização de forma mais geral pode ser tracejada pelo
menos até a invenção do cálculo diferencial de forma independente por Isaac
Newton e Gotfried Leibiniz entre os séculos XVII e XVIII, cujos métodos poderiam ser usados,
apesar da pouca viabilidade, para otimizar o valor de uma função contínua. No decorrer
do tempo, o estudo do cálculo e, consequentemente, dos métodos de otimização se
expadiram com os trabalhos de outros matemáticos como Leonard Euler, Adrien-Marie
Legendre e Friedrich Gauss.
Diversos métodos foram desenvolvidos ao longo dos séculos por esses matemáticos,
porém muitos com a propriedade de serem inviáveis de serem aplicados em larga
escala, como os Multiplicadores de Lagrange.

Os problemas de programação matemática, e mais ainda os de programação linear,
surgem naturalmente em uma série de aplicações, porém,
durante e após a Segunda Guerra Mundial, a resolução desses problemas
tornou-se crítica. A razão é que num conflito daquela escala
problemas da natureza de "gerir e alocar recursos
escassos de maneira eficiente" tornaram-se uma das principais preocupações dos
militares da época.

Provavelmente o primeiro a formular explicitamente a teoria e problemas de
programação linear foi o matemático soviético Leonid Kantorovich (1912-1986)
enquanto trabalhava em
problemas de planejamento e gestão de recursos para os militares da URSS na
década de 1930. Contudo, os trabalhos de Kantorovich permaneceram desconhecidos
no Ocidente até o final
dos anos de 1950 devido à Cortina de Ferro da Guerra Fria.

A historiografia da Programação Linear geralmente atribui como marco de seu início o
desenvolvimento do método simplex em 1947 pelo matemático norte-americano
George B. Dantzig (1914-2005), que o criou como parte dos esforços do
\textit{Scientific Computation Optimum Program} (SCOOP) da força aérea dos
Estados Unidos. O termo "programação linear" é atribuído ao matemático
americano T. C. Koopmans, em que a palavra programação não significava
escrever um programa de computador, estando na verdade atrelada ao sentido de
"planejamento" de uma solução.

Desde a invenção do simplex, a teoria e os algoritmos, não só de programação
linear, mas da otimização matemática de forma geral, foram desenvolvidos e
expandidos por uma série de autores ao longo do século XX. Suas aplicações
espalharam-se para diversos campos, como a própria matemática, a pesquisa
operacional, a economia, estatística e diversas outras ciências. Os problemas
de PL também impulsionaram o desenvolvimento de métodos para problemas de maior
complexidade, como a programação discreta, a programação não linear, a
programação combinatória, programação estocástica, problemas de controle ótimo,
etc.
